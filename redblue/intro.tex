\section{Motivation and contributions}
\label{ch:redblue:sect:motiv}
As we mentioned in Chapter~\ref{chapter:intro}, scaling services over the Internet to meet the needs of an
ever-growing user base is challenging. In particular, in order to improve
user-perceived latency, which directly affects the quality of the user
experience, services replicate system state across geographically
diverse sites and direct users to the closest or least loaded site.


To avoid paying the performance penalty of synchronizing concurrent
actions across data centers, some systems, such as Amazon's
Dynamo~\cite{Decandia2007Dynamo}, resort to weaker consistency semantics
like eventual consistency where state can temporarily diverge.
Others, such as Yahoo!'s PNUTS~\cite{Cooper2008PNUTS}, avoid state
divergence due to the undesirable sets of behaviors it allows, by
requiring all \operations\ that update the service state to be
funneled through a primary site and thus incurring increased latency.

In order to address the inherent tension between improving performance and
maintaining meaningful consistency semantics, several approaches have been recently proposals 
for allowing multiple
  levels of consistency to
  coexist~\cite{Ladin1992LazyReplication,Sovran2011PSI,Singh2009Zeno}: some
  \operations\ can be executed optimistically, without synchronizing
  with concurrent actions at other sites, while others require a
  stronger consistency level and thus require cross-site
  synchronization. However, this places a high burden on the developer
  of the service, who must decide which \operations\ to assign which
  consistency levels. It is challenging to make such decisions since it requires reasoning about the consistency
  semantics of the overall system to ensure that the behaviors
  that are allowed by the different consistency levels satisfy the specification
  of the system.


In this chapter we present a comprehensive and principled approach to
this problem, aiming at enabling geo-replicated systems to be as fast
as possible, while ensuring that they are consistent when necessary. We make the following three contributions:
\begin{enumerate}
\item 
We propose a novel consistency definition called \RBc. The intuition
behind \RBc\ is that \blue\ operations execute locally and are lazily
replicated in an eventually consistent manner~\cite{Decandia2007Dynamo,Lloyd2011Causal,Terry1995Managing,
Mahajan2010Depot,Feldman2010Sporc,Shapiro2011Conflict,Singh2009Zeno}.\ \Red\ \operations, in contrast, are serialized with respect to each
other and require immediate cross-site coordination. In addition,
\RBc\ preserves
causality by ensuring that dependencies established when an
\operation\ is  invoked at its primary site are preserved as
the \operation\ is incorporated at other sites.

\item We identify the sufficient conditions under which \operations\ must be
  colored \red\ and may be colored \blue\ in order to ensure that
  application invariants are never violated and that all replicas
  converge on the same final state.  Intuitively, \operations\ that
  commute with all other \operations\ and do not impact invariants may
  be \blue; the remaining ones must be \red.

%an \operation\ may be \blue\ if it commutes with all other
%\operations\ and does not impact any invariants.

\item We observe that the commutativity requirement limits the space
  of potentially \blue\ \operations, provided that many \operations\ in real world applications
do not commute w.r.t each other. To address this limitation, we decompose \operations\ into two
  components: (1) a \initial\ \operation\ that identifies the changes
  the original \operation\ should make, but has no side effects
  itself, and (2) a \shadow\ \operation\ that performs the identified
  changes and is replicated to all sites. With this decomposition, only
  \shadow\ \operations\ are colored \red\ or \blue. This allows for a
  dynamic runtime classification of \operations\ and hence broadens the space
  of potentially \blue\ \operations.
\end{enumerate}

We built a system called \gemini\ that coordinates
\RedBlue\ replication, and use it to extend three applications to be
\RBct: the TPC-W and RUBiS benchmarks and the Quoddy social network.  Our evaluation using microbenchmarks and the three
applications shows that \RBc\ provides substantial latency and
throughput benefits.

