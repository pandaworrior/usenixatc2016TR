\section{System model}
\label{chapter:model}
 We assume a distributed system with state fully rep\-licated across
 $k$ sites denoted $\textit{site}_0\ldots \textit{site}_{k-1}$.  We
 follow the traditional deterministic state machine model, where there
 is a set of possible states $\mathcal{S}$ and a set of possible
 \operations\ $\mathcal{O}$, each replica holds a copy of the current
 system state, and upon applying an \operation\ each replica
 deterministically transitions to the next state and possibly outputs
 a corresponding reply.  \allen{expand on the following in a TR}
 \changebars{}{For uniformity, we treat the output as a separate part
   of the replica state.}

In our notation, $S\in\mathcal{S}$ denotes a system state, and
$u,v\in\mathcal{O}$ denote operations.\ We assume there
exists an initial state $S_0$.  If operation $u$ is applied against
a system state $S$, it produces another system state $S'$;
%jg10 changed this
we will also denote this by $S'=S+u$.
%which we denote by $S+u=S'$. 
We say that a pair of \transactions\ $u$ and $v$ 
{\em commute} 
%jg10: Definitions do not need an iff
if $\forall S \in \mathcal{S},S+u+v=S+v+u$.
%jg10: Do not need ``In addition''
%In addition, t
Given a total order $T(U, <)$ over a
set of operations $U$, if we apply
all operations $U$ against a system state $S$ according
to $<$, we denote the final state by $S(T)$. $S(T) = $
$S+u_{0} + u_{1} + ... + u_{i} + ... + u_{|U|-1}$, where $\forall i, 0<=i<|U|$. 
The system maintains a set of application-specific
invariants. We define the primitive $\textit{valid}(S)$ to be {\it
  true} if state $S$ satisfies all these invariants and {\it false}
otherwise. We say an operation $u$ is {\it correct} if for any
valid state $S$, $S+u$ is also valid.
%jg10: Do not need ``Finally''
%Finally, 
%jg10: Also changed this sentence
Each \operation\ $u$ is initially submitted at one site which we
call $u$'s \emph{primary site} and denote $\textit{site}(u)$;
%jg10: Added this sentence
the system then later replicates $u$ to the other sites.
%Each \operation\ $u$ is associated with a unique primary site
%$\textit{site}(u)$ where $u$ was submitted.


\allen{We could drop this paragraph for space --- the discussion is
  now in the architecture section.}  
%new
\changebars{} 
%old
{
%\paragraph
{ Failure modes} In the description of our new consistency model and
of the implementation of the system, for simplicity, we assume that
individual nodes obey their expected behaviors, and we do not consider
node crashes or network failures.  Our focus in this paper is in
understanding what should be the specification for a replicated system
that works well when concurrent \operations\ are executed by clients
in different sites, and how to make the best possible use of that
specification.  We present a brief discussion on how to implement that
specification in a fault-tolerant way in \S \ref{sect:handleFaults}.
}
%done







