
\vspace*{\stretch{1}}
\section*{Abstract}

Distributing data across replicas within a data center or across multiple data centers
plays an important role in building Internet-scale services that provide
good user experience, namely low latency access and high throughput. This approach
often compromises on strong consistency semantics, which help maintain
application-specific desired properties, namely, state convergence and invariant preservation. 
To relieve such inherent tension, in the past few years, many proposals have been
designed to allow programmers to selectively weaken consistency levels of
certain operations to avoid costly immediate coordination for concurrent user requests. However,
these fail to provide principles to guide programmers to make a correct decision
of assigning consistency levels to various operations so that
good performance is extracted while the system behavior still complies with its specification.
% Others leave a non-trivial
%burden on programmers of changing or designing their services to adopt
%a specific consistency model.}

The primary goal of my work is to provide programmers with principles and tools for building
fast and consistent systems by allowing programmers
to think about various consistency levels in the same framework. The first step we
took was to propose \RBCN, which presents sufficient
conditions that allow programmers to safely separate weakly consistent operations from
strongly consistent ones in a coarse-grained manner. Second, to improve the practicality of \RBCN,
we built SIEVE - a tool that explores
both Commutative Replicated Data Types and program analysis techniques to assign proper consistency levels
to different operations and to maximize the weakly consistent operation space.
Finally, we generalized the tradeoff  between consistency and performance and proposed \PRCNF\ (or short, \PRCN) - a generic
consistency definition that captures various consistency levels in terms of visibility restrictions
among pairs of operations and allows programmers to tune the restrictions to obtain a fine-grained control of
their targeted consistency semantics.


\vspace*{\stretch{1}}

\cleardoublepage
\vspace*{\stretch{1}}
\section*{Kurzdarstellung}
Daten auf mehrere Repliken in einem Datenzentrum oder 
{\" u}ber mehrere Datenzentren zu verteilen, nimmt einen hohen Stellenwert ein, um 
Internet-weite Services mit guter Nutzererfahrung, insbesondere mit niedrigen 
Zugriffszeiten und hohem Datendurchsatz, zu implementieren. Diese Methode beeintr{\"a}chtigt in der 
Regel die starke Konsitenzsemantik, die hilft gew{\"u}nschte anwendungsspezifische Eigenschaften, die 
Zustandskonvergenz und Erhaltung von Invarianten, aufrechtzuerhalten. Um diesen Kompromiss zu mildern, 
wurde in den letzten Jahren mehrere Vorschl{\"a}ge entworfen, die es dem Programmierer erm{\"o}glichen f{\"ur} 
einzelne Operationen ein schw{\"a}cheres Konsitenzlevel auszuw{\"a}hlen, um der aufwendigen Koordination paralleler 
Benutzeranfragen zu entgehen. Allerdings liefern diese Leits{\"a}tze f{\"u}r die Programmierer keine L{\"o}sungsans{\"a}tze, wann 
welches Konsistenzlevel f{\"u}r eine Operation anzuwenden ist, so dass die h{\"o}chstm{\"o}gliche Leistung erreicht wird und 
gleichzeitig die Handlung des Systems die Spezifikation erf{\"u}llen.

Das Hauptziel dieser Doktorarbeit ist es Leits{\"a}tzen und Werkzeuge f{\"u}r Programmierer bereitzustellen, 
die die Entwicklung von leistungsstarken und konsistenten Sytemen erm{\"o}glichen, in dem dem Programmierer mit Hilfe eines 
Frameworks gleichzeitig zwischen verschiedenen Konsistenzlevel w{\"a}hlen kann. Als ersten Schritt entwickelten wir RedBlue Konsistenz, 
welches die hinreichende Bedingungen erl{\"a}utert, die es einem Programmierer erlauben zwischen schwacher Konsistenz und starker 
Konsistenz zu w{\"a}hlen. Um die Praktikabilit{\"a}t von RedBlue Konsistenz im zweiten Schritt weiter zu erh{\"o}hen, entwickelten wir 
SIEVE - ein Werkzeug, das sowohl kommutative, replizierte Datentypen und Programmanalyseverfahren verwendet, um den richtigen 
Konsistenzlevel zu verschiedenen Operationen zuzuordnen und dabei die schwach konsistenten Operationen zu maximieren. 
Abschliessend verallgemeinern wir den Kompromiss zwischen Konsistenz und Leistungsst{\"a}rke und stellen die partiell, 
eingeschr{\"a}nkt geordnete Konsistenz vor (PoR Konsistenz) - eine generische Konsistenzdefinition, die verschiedene 
Konsistenz level, hinsichtlich der Einschr{\"a}nkung der Sichtbarkeit zwischen paaren von Operationen, umfasst und dem 
Programmierer erlaubt, die Einschr{\"a}nkungen zu justieren, um die gew{\"u}nschte Konsistenzsemantik zu erzielen. 
\vspace*{\stretch{1}}
\cleardoublepage
