\chapter{Conclusion}
\label{chapter:conclude}

In the recent few years, (geo-)replication has been widely adopted to build scalable services
that offer low latency access and high throughput, in order to meet
their unprecedented user demands. However,
this goal is often compromised by the coordination required to 
ensure application-specific properties such as state convergence and invariant preservation.
This dissertation shows that differentiating the consistency requirements for various operations
and executing operations with different amount of coordination can make
replicated services fast as possible while ensuring their targeted consistency semantics. 

In short, our approach consists of the following three major components: a) \RBCN, a novel
consistency offers a coarse-grained choice between executing an operation under either
strong consistency or weak (causal) consistency; b) \tool,
a tool automatically makes an decision on which consistency level to be assigned to any
operation in the context of \RBCN; and c) \PRCN,
another novel consistency definition generalizes the tradeoffs behind \RBCN, offers a fine-grained
choice in consistency requirements for various operations and reduces the amount of
required coordination to the greatest extent possible.

\RBCN\ allows strongly and weakly consistent operations to coexist
in a single system and defines a set of sufficient conditions to determine the proper
consistency levels for various operations by analyzing
whether running operations in parallel can make state diverge or be invalid. In short, 
an operation must be \red\ (strongly consistent) if either it does not commute with any other operation or
it potentially breaks invariants in the presence of concurrency; otherwise, it can be \blue\ (causally consistent). 
To address the problem that many original operations do not naturally commute,
we propose to decouple the execution of an operation into a generator operation to decide
the changes but has no side effects, and a shadow operation to apply identified changes in
a commutative fashion across all replicas. At end, we built \gemini, which is 
a distributed coordination and replication tier for making web applications \RBct.

To the best of our knowledge, \tool\ is the first system to automate the
choice of consistency levels offered by \RBCN\ in a replicated system. 
It relieves the programmer from having to a) construct commutative shadow operations;
and b) reason about the behaviors that weak consistency introduces, only requiring the
programmer to write the system invariants that must be preserved
and provide a small amount of annotations regarding merge semantics. To automate
a) step, we leverage CRDTs to translate every SQL statement into commutative forms. 
To automate b) step, we reply on weakest precondition analysis
techniques to summarize sufficient logic conditions, under which the corresponding shadow operations
can be invariant-safe. At runtime, an efficient evaluation of such conditions will
tell whether strong consistency or weak consistency must or can be used.

\PRCN\ has a broader view of the tradeoffs between maintaining targeted consistency semantics and
improving performance as possible by expressing these semantics into visibility restrictions
between pairs of operations. Weakening or strengthening the consistency semantics in the
context of \PRCN\ is achieved by imposing fewer or
more restrictions over relevant operations. In order to minimize the amount of required coordination,
we developed a concept called {\tt I-conflict set}, which captures the finest composition of
shadow operations corresponding to an invariant violation, and  a sequence of algorithms to explore
{\tt I-conflict} sets and add the relevant restrictions. Finally, to help programmers make use of
\PRCN, we built an efficient coordination service called \coordtool, which allows the programmer
to choose the most lightweight protocol for replicating operations while serializing
any pair of operations that need to be restricted.

In summary, there exists some limitations in the dissertation work, but we hope that our approach 
and all choices behind it will provide a useful reference for the system designer of either distributed
systems or multi-core concurrent systems.
