\section{Generating shadow operations}
\label{ch:sieve:sect:commute}

This section covers how we automate the conversion of application code into commutative \emph{shadow operations}.

%JLEITAO: Review text of the table later.
\if 0
\begin{table*}[t!]
%\footnotesize
\centering
\begin{tabular}{|c|c|p{8cm}|}
\hline
Abstract types & Inherited types & Description \\
\hline
NUMDELTA & Integer, Double, Float, DateTime &
  Its computation is either increment or decrement. Its commutative solution
  is to apply a delta with the same type (either negative or positive) to itself.  \\ \hline
LWW & Integer, Double, Float, DateTime, String &
  LWW stands for last-writer-win. Among a set of concurrent operations
  for the same data, only one of them will be deterministically selected
  to install the final value. The selection depends on a total order.\\ \hline
\multirow{4}{*}{SET} & AOSET & It is an add-only set, which only supports disjoint and non-concurrent
          insertions.\\
 \cline{2-3}
&  UOSET & It is an update-only set, which only support concurrent updates.\\
 \cline{2-3}
&  AUSET & It is an add-and-update set, which supports unique insertion and concurrent updates.\\
 \cline{2-3}
&  ARSET & It is an add-and-remove set, which support concurrent insertions, updates
            and deletes. Among concurrent insertions of the same data, only one insertion coming first will
            succeed, and the rest of them will be translated into updates. Among concurrent updates of the same
            data, only one will be deterministically selected to install the final value. Among concurrent
            deletions of the same data, only one with highest ordering information will eventually remove the
            data.\\
\hline
\end{tabular}
\caption{Commutative replicated data types (CRDTs) supported by our type system}
\label{tab:crdts}
\end{table*}
\fi

%\subsection{Static component: leveraging CRDTs}
\subsection{Leveraging CRDTs}

% paragraph I - you are here, let's use DBs and CRDTs.
%As mentioned in the previous section, w
We leverage several observations
and technologies to achieve a sweet spot between the need to capture the
semantics of the original operation when encoding its side effects and
the desire to minimize the amount of programmer intervention.
First, we observe that many applications are
built under a two-tier model, where all the persistent state of the
service is stored in a relational database accessed through SQL commands.
Second, we leverage CRDTs~\cite{Preguica2009CRDT}, which
%, like other
%operational transformation techniques~\cite{Sovran2011TranGeo}, % RIGHT REF: SPORC 
construct operations that commute by design by encapsulating all side
effects into a library of commutative operations.
%in the literature, such as CRDTs~\cite{Preguica2009CRDT}, %Cset~\cite{Sovran2011TranGeo},
%and Z-relation~\cite{Green09Reconcilable}. To transform transactions into shadow operations
%we rely on CRDTs, and in particular we have implemented our own CRDT library that both
%incorporates ideas of existing proposals and explicitly takes into account the typical needs of
%the web applications. For completeness we present in Table~\ref{tab:crdts} the CRDTs 
%types supported by our library.

% II These two allow you to overcome one obstacle of CRDTs: rewriting code
These two concepts allow us to achieve commutativity while overcoming
the disadvantage of CRDTs, namely the need to adapt applications.
This is because the state of two-tier applications is
accessed through the narrow SQL interface, and therefore we can focus
exclusively on adapting the implementation of SQL commands to access a
CRDT. In particular, database tables can be seen as a set of tuples, and
therefore all the calls in the original operation to add or remove
tuples in a table can be replaced in the shadow operation with a CRDT set
add or remove, which, in turn, is implemented on top of the
database. 

%The programmer only has to select the appropriate merging
%strategy (i.e., the adequate CRDT type) to encode these operations, without
%being required to program these CRDT transformations or to change the
%code of each operation.

%JLEITAO: Review text of the table later.
% \begin{table}[t!]
% \footnotesize
% \centering
% \begin{tabular}{|p{1.5cm}|p{2cm}|p{3cm}|}
% \hline
% Abstract type & Inherited type & Description \\
% \hline
% LWW (FIELD) & Integer, Double, Float, DateTime, String&  Uses last-writer-wins to solve concurrent updates\\\hline
% NUMDELTA (FIELD) & Integer, Double, Float, DateTime  &  Add a delta to the value \\\hline
% \multirow{4}{*}{SET} & AOSET,                       UOSET,                AUSET,              ARSET & For concurrent operations,
% higest timestamp wins. Insert is translated to an
% update if the record exists.\\
% \hline
% \end{tabular}
% \caption{Commutative replicated data types (CRDTs) supported by our type system.}
% \label{tab:crdts}
% \end{table}

% \begin{table}[t!]
% \footnotesize
% \centering
% \begin{tabular}{|p{1.5cm}|p{2cm}|p{3cm}|}
% \hline
% SQL type & CRDT & Description \\
% \hline
% \multirow{3}{*}{FIELD*} & \multirow{2}{*}{LWW} &  Use last-writer-wins to solve concurrent updates\\\cline{2-3}
%  & \multirow{2}{*}{NUMDELTA}  &  Add a delta to the numeric value \\\hline
% \multirow{4}{*}{TABLE} & AOSET,                       UOSET,                AUSET,              ARSET & Declare accepted mutations
% for a table, and merge concurrent inserts, updates or deletes on demand\\
% \hline
% \end{tabular}
% \caption{Commutative replicated data types (CRDTs) supported by our type system. *
% FIELD covers primitive types such as integer, float, double, datetime and string.}
% \label{tab:crdts}
% \end{table}


\begin{table}[t!]
%\small
\centering
\begin{tabular}{|>{\centering\arraybackslash}m{1.2in}|>{\centering\arraybackslash}m{0.8in}|>{\centering\arraybackslash}m{1.9in}|}
\hline
SQL type & CRDT & Description \\
\hline
\hline
\multirow{3}{*}{FIELD*} & LWW &  Use last-writer-wins to solve concurrent updates\\\cline{2-3}
 & NUMDELTA  &  Add a delta to the numeric value \\\hline
TABLE & AOSET,                       UOSET,                AUSET,              ARSET & Sets with restricted operations (add, update, and/or remove).
Conflicting ops.\ are logically executed by timestamp order.\\
\hline
\end{tabular}
\caption{Commutative replicated data types (CRDTs) supported by our type system. *
FIELD covers primitive types such as integer, float, double, datetime and string.}
\label{tab:crdts}
\end{table}

% III However, there is some semantic that needs programmer - eg
However, it is impossible to completely remove the programmer from the
loop, due to the choice of which CRDT to use for encoding appropriate merging
semantics. For instance, when an integer field of a tuple is written
to in a SQL update command, the programmer could have two different
intentions in terms of what the update means and how concurrent
updates should be handled: (1) the update can represent a delta to be
added or subtracted from the current value (e.g., when updating the
stock of a certain item), in which case all concurrent updates should
be applied possibly in a different order at all replicas to ensure that no stock
changes are lost, or (2) it can be overwriting an old value with a new
value (e.g., when updating the year of birth in a user profile), in
which case an order for these updates should be arbitrated, and the last
written value should prevail. Even though both strategies ensure
state convergence, their semantics differ significantly. For example,
the second strategy leads to a final state that does not reflect the effects
of all update operations.
%In CRDT terms, this semantic information
%is encoded as the use of different data types.

% IV We address this using simple annotations to schema.
% V - some details, not too many please
Since the appropriate merging strategy is application-specific,
the programmer has to convey this decision.
To minimize this input, we only require the programmer to select the appropriate merging
strategy (i.e., the adequate CRDT type) to encode these operations rather than programing these CRDT transformations or changing the
code of each operation. In more detail, we provide programmers a number of CRDT types (shown in Table~\ref{tab:crdts}),
and they should declare which types to use on a per-table and per-attribute basis. These
types form two categories: field, which is the smallest component of 
a record and defines its commuting update operation in the presence of
concurrency, and set, which is a collection of such records plus the support 
for commutative appending or removing. Programmers only need to annotate 
the data schema with the desired CRDT type using the following
annotation syntax:

\begin{equation} 
@[CRDT Name] [Table Name | Data Field Name]
\end{equation} 

\begin{figure}[t]
\centerline{\small\begin{tabular}{l}
\sf  \underline{\bf@AUSET} CREATE TABLE exampleTable (
\\\sf~~~~  objId INT(11) NOT NULL,
\\\sf~~~~  \underline{\bf@NUMDELTA} objCount INT(11) default 0,
\\\sf~~~~  \underline{\bf@LWW} objName char(60) default NULL,
\\\sf~~~~  PRIMARY KEY (id)
\\) ENGINE=InnoDB
\end{tabular}}
\caption{Annotated table definition schema.}
\label{fig:annoSql}
\end{figure}

Figure~\ref{fig:annoSql} presents a sample annotated SQL table creation
statement. 
We assign \texttt{exampleTable} the type \texttt{AUSET} (Append-Update Set),
a CRDT set that only allows append and update operations, thus
precluding the concurrent insertion and deletion of the same item %, which require
%semantic information to merge 
(less restrictive CRDT sets also exist).
The field \texttt{objCount} associated with \texttt{NUMDELTA} always
expects a delta value to be added or subtracted to its current value. 
By default, if no annotations are provided, we conservatively
mark the corresponding table or 
field to be read-only.

\subsection{Runtime creation of shadow operations}

With these schema annotations in place, it is easy to generate
commutative shadow operations at runtime. The idea is to invoke the
original operation upon the arrival of a new user request (as would
happen in a system that does not make use of shadow operations) but
with the difference that all the calls to execute commands in the
database are intercepted by a modified JDBC driver that builds the
sequence of CRDT operations that comprise the shadow operation as the
original operation progresses. Furthermore, using the schema
annotations, \tool\ maps each database update to an
appropriate merge semantics and replaces the operations on a certain table or field
with the appropriate operations over the corresponding CRDT type.

For instance, to create a shadow operation for a transaction that
updates \texttt{objCount} in Figure~\ref{fig:annoSql},
when an update is invoked, we first query the old value $s$,
and then, given  the new value
$s'$, we compute a \texttt{delta} by subtracting $s$ from
$s'$. Finally, we use \texttt{delta} and the primary key $pk$ of
the corresponding object to parameterize
a CRDT operation that reads the tuple
identified by $pk$ and then adds \texttt{delta} to it.
%. When replicating this shadow
%operation, we always adds \texttt{delta} to the database
%record specified by $pk$.

% to automatically transform the side effects of the application
% code into shadow operations that are commutative by design.

%To enable the automatic transformation of transactions into (commutative) shadow operations, we require
%programmers to provide simple annotations over the database schema, which indicate which CRDS type
%should be used to manipulate each data object. 

\if 0
Additionally, to ensure that shadow operations are commutative we also eliminate all sources
of non-determinism that might exist in the operation code, which could lead to state divergence
when executing the shadow operations. This is achieved by transparently encoding deterministic
values into CRDTs whenever sources of non-determinism are used. For example, if some transaction relies
on the current time, we simply replace the call that provides this value, with a static value obtained
at the creation of the shadow operation. If an updating operation does not specify the primary key, 
we first perform a select to fetch a list of primary keys from the set of records matching the
condition, and then encode the primary keys along the updating semantics.
\fi

%% {\em [There are here some details that distinguish between the conversion in the static component
%% of the tool, and at run time. However it is unclear if at this point the reader already dominates these 
%% concepts enough to be able to understand this. -- Must be solved!!!]}
%% The only
%% difference between the runtime and static fronts is that the static transformation
%% translates a snippet of code into a sequence of CRDT operations, while the counterpart
%% at runtime intercepts every SQL statement execution,

Finally, when the original operation issues a commit to the database,
the tool outputs a shadow operation containing the accumulated
sequence of CRDT operations.

%% \begin{figure*}[b!]
%% \begin{subfigure}[Original sql schema]{
%% \begin{minipage}[b]{0.5\textwidth}
%% \lstinputlisting{./code/sqlSchema.txt}%                                                                                                                                                          
%% \label{fig:originalSql}
%% \end{minipage}
%% }
%% \end{subfigure}
%% \hspace{0.4cm}
%% \begin{subfigure}[Annotated table definition schema]{
%% \begin{minipage}[b]{0.5\textwidth}
%% \lstinputlisting{./code/annoSqlSchema.txt}%                                                                                                                                                      
%% \label{fig:annoSql}
%% \end{minipage}
%% }
%% \end{subfigure}

%% \caption{Annotation example.}
%% \label{fig:annoSqlExample}
%% \end{figure*}

\subsection{Miscellaneous}

In this part, we discuss a few interesting aspects related to the commutativity
conversion.

\paragraph{Treatment of non-deterministic SQL statements.} 
In addition to the previously described logic to construct commutative shadow operations,
we also eliminate all sources of non-determinism that might exist in the operation code, which could lead to state divergence
when executing the shadow operations. This is achieved by transparently encoding deterministic
values into CRDTs whenever sources of non-determinism are used. For example, if some transaction relies
on the current time, we simply replace the call that provides this value, with a static value obtained
at the creation of the shadow operation. Some queries update or delete
records when these records match a certain condition rather than specifying the primary keys.
With regard to this case, we first perform a select to fetch a list of primary keys from the set of records matching the
condition, and then encode the primary keys along the updating semantics.

%There exist a few non-deterministic
%updating queries, which may introduce different side effects if they are applied
%against different system states. The simplest example is an update query
%invoking database functions to get non-deterministic inputs, such as $NOW()$,
%$CURRENT\_TIMESTAMP$ and etc. More complex, an UPDATE query embedded with a
%SELECT statement changes state with the value read by the SELECT. One another
%example is to delete or update all records from a table where these records
%match a certain condition. In these two cases, we have to first change these
%queries into deterministic. To do so, we force the transaction to fetch the
%non-deterministic state before performing updates, and the non-deterministic
%state will be encoded into the corresponding shadow operations. For the update
%example, the non-deterministic state must be the state read from the embedded
%selection. Regarding the delete or update example, the non-deterministic state
%is the list of records matching the specified condition when the transaction is
%executed.

%\paragraph{Unique identifier generation.} {\tt auto\_increment} is
%a powerful and wide-deployed database built-in technique for generating
%unique identifiers without invoking an additional selection to determine
%the next available identifier. This technique explicitly define an invariant that no
%two identifiers are identical. To maintain this invariant, we have to label as red
%all shadow operations in which a newly inserted record having an {\tt auto\_increment} attribute. 
%To overcome this problem, we provide a
%generic unique id generation function, which will automatically produce globally
%unique id without coordination with other sites. 

\paragraph{Annotation suggestion.} We found that it would be possible to
effectively recommend a few commutative types to programmers, by statically
analyzing the application code, or the SQL queries, or both. For example, if a
data field is always modified by an assignment statement, then a $LWW$ solution
may be suitable for it. If a data field is manipulated via either addition or
subtraction, then the $NUMDELTA$ is a good CRDT candidate. Furthermore, if a table is
never modified by a delete query, then we would suggest it to be tagged as an
AUSET (Add-Only set). We leave this optimization to our future work.

