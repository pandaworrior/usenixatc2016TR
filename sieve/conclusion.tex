\section{Limitations and future work}
\label{ch:sieve:sect:limitation}
Although \tool\ reduces human intervention that might be involved in making the choice of
CRDTs and consistency levels for scaling out web services, there are still several points for optimization, 
which we leave as future work to address.

First, while the CRDT library covers a set of
most representative CRDT types that suffice for all use cases exhibited in our case study applications,
it does not include some more recent proposals like maps~\cite{riakmap},
and does not have a full coverage of SQL features defined in ~\cite{SQLStandard}. This incompleteness
may limit the selection of merging semantics, which the programmer may require to use not only for ensuring
state convergence, but also for providing meaningful merged outcomes.

Second, we observed performance degradation
when running unreplicated applications with \tool. This is because we implemented the CRDT
transformation in a JDBC driver and it requires us to parse every SQL statement to figure out
the side effects. One possibility is augmenting the database code with this logic so that
we can take advantage of rich information from query execution plans generated by the database.

Third, our approach is based on a fundamental assumption that 
iterations in loops are independent w.r.t each other, 
so that weakest preconditions can be efficiently computed. This assumption is also 
a limitation of our approach, as \tool\ will conservatively generate a {\tt FALSE} condition
for operations if their precondition computation fails, in case such a loop independence
property does not hold. Additionally,
we would like to explore algorithms to automatically verify loop independence,
instead of relying on manual processing.

\section{Conclusion}
\label{ch:sieve:sect:conclude}

In this chapter, we presented \tool, which is, to the best of our knowledge, 
the first tool to automate the choice of consistency levels in a replicated system. Our
system relieves the programmer from having to reason about the
behaviors that weak consistency introduces. \tool\ minimizes
human intervention by only requiring the
programmer to write the system invariants that must be preserved
and to provide annotations regarding merge semantics. Our
evaluation shows that \tool\ labels operations accurately, incurring a modest
runtime overhead when compared to labeling operations manually and
offline.
