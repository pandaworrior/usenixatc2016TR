\chapter{System model}
\label{chapter:sysmodel}

%\section{System model}\label{sect:model}
%\pagelimit{0.5}
In this chapter, we present the system model our work is built atop and
a set of desirable end-to-end system properties (marked in bold) we intend to provide.

We align our notations with those defined in the well-known state
machine replication literature~\cite{Schneider1990RSM}. We assume a distributed system with state 
fully replicated across $k$ sites denoted by $\textit{site}_0\ldots \textit{site}_{k-1}$. Each site
hosts a replica, and each replica behaves following a deterministic state machine model.
It is worth mentioning that site is a logical unit that hosts a full copy of system state,
and hence it is possible to have multiple sites across geo-graphically dispersed data
centers or even within a single data center. In the rest of the document, the terms ``site'' and ``replica'' are interchangeable.

The system defines a set of operations $\mathcal{U}$ manipulating
a set of reachable states $\mathcal{S}$. We do not restrict the type of operations that
can be executed within that system, a property we call {\bf general operations}.
If operation $u$ is applied against a system state $S$, 
it produces another system state $S'$; we will also denote this by $S'=S+u$. 
Given a total order $T(U, <)$ over a
set of operations $U$, where $U \subset \mathcal{U}$, if we sequentially apply
all operations $U$ against a system state $S$ according
to $<$, then we denote the final state by $S(T)$. $S(T) = $
$S+u_{0} + u_{1} + ... + u_{i} + ... + u_{|U|-1}$,
where $0\leq i<|U|$. We say that a pair of operations $u$ and $v$ {\em commute} 
if $\forall S \in \mathcal{S},S+u+v=S+v+u$. An operation $u$
is {\it globally commutative}, if it commutes
with all operations in $\mathcal{U}$ (including itself).

Each operation $u$ is initially submitted by a client at one site which we
call $u$'s \emph{primary site} and denote $\textit{site}(u)$;
the system then later replicates $u$ to all remaining sites. Upon receiving an operation, replicas
at the recipient sites apply it against their local state. It is important that 
all replicas that have executed the same set of operations are in the same state, 
i.e., that the underlying system offers a {\bf state convergence} property; 
otherwise a quiescent system would return different views of the
state depending on which replicas the users connected to.
%; and the system should provide
%a set of {\bf stable histories}, meaning that user actions cannot be undone.

As clients are always expecting fast responses to their requests, 
we aim to provide {\bf low latency} access to the service~\cite{Schurman2009latency}. 
Another important property that consists of ensuring a good user
experience is to preserve \textbf{causality}, both in terms of the monotonicity of user requests
within a session and preserving causality across clients, which is key
to enabling natural semantics~\cite{Petersen1997Flexible}. Additionally,
operations invoked by client requests should return a {\bf single value}, precluding solutions that return a set of values
corresponding to the outcome of multiple concurrent updates.

\if 0
If such a system received
a set of operations $U$ and generated corresponding replies to the operation initiators (clients),
then all replicas must eventually receive all operations in $U$, when the system becomes
quiescent, i.e., no more operations are coming and all operations are reliably propagated.
Every replica then executes $U$ in a sequence, while different replicas might observe
different sequential execution orders.
\fi

The system also maintains a set of application-specific
{\bf invariants}. For instance, in a banking application,
bank balance values are never negative; stock values must be non-negative as well in a shopping
cart application; and, in a bidding website, the winner of an auction must issue
the highest accepted bid. To capture this, we define the primitive $\textit{valid}(S)$ to be {\it
  true} if state $S$ satisfies all these invariants and {\it false}
otherwise. We denote a valid initial state of every service by $S_0$. We say an operation $u$ is {\it correct} if for
all valid states $S$, $S + u$ is also valid. In the previous banking example,
a {\tt deposit} operation, adding a positive delta to a user's account balance,
is {\it correct}, as its application against any valid state always
ensures that the corresponding balance value is above zero. Unlike
{\tt deposit}, a {\tt withdraw} operation can be {\it correct} if it includes
an {\tt if} statement to check whether there is enough balance,
 but would be incorrect if the programmer did not include this check.





