\subsection{State convergence}
\label{sec:properties:converge}
A PoR-consistent replicated system is state convergent if all its replicas reach
the same final state when the system becomes quiescent, i.e., for any pair of causal legal serializations
of any R-Order, $L_{1}$ and $L_{2}$, we have $S_0(L_1) = S_0(L_2)$, where $S_0$ is the valid initial state. We
state the necessary and sufficient conditions to achieve this in the following theorem.

%\begin{mydef}
%\textbf{Convergence}: A system $\mathscr{S}$ starting from an initial state $S$ and with a set of restrictions $R$ is convergent if,
%for any execution of $\mathscr{S}$ and its corresponding R-Order $O$, any two legal serializations of $O$, $L_{1}$ and $L_{2}$, are state convergent, i.e., $S(L_{1}) = S(L_{2})$.
%\label{def:converge}
%\end{mydef}

%\cheng{Why we use the term legal serialization instead of causal legal serialization is because that the number of replicas can be dynamically changed.}

\begin{theorem}
\label{theorem:convergence}
\emph{(Convergence Theorem)}
A \PRCAJ\ system $\mathscr{S}$ with a set of restrictions $R$ is \textbf{convergent}, if and only if,
for any pair of its shadow operations $u$ and $v$, $r(u, v) \in R$ if $u$ and $v$ don't commute.
\end{theorem}

In order to prove this theorem, we need the assistance from three lemmas introduced in Chapter~\ref{chapter:redblue}, namely,
Lemmas~\ref{lem:canSwap}, ~\ref{lem:adjexists} and ~\ref{lem:adjacentconvergent}. The first lemma asserts that,                             
given a legal serialization, swapping two adjacent
shadow operations in this serialization that are not ordered by the R-Order
results in another legal serialization. The second lemma asserts that given an R-Order and one of its legal serializations, if there exists
a pair of shadow operations $u$ and $v$ that are not ordered by the R-Order, then there exists an adjacent
pair of shadow operations between $u$ and $v$ in that serialization that are not ordered
by the R-Order. The third lemma asserts that two legal serializations that differ
in the order of exactly one pair of adjacent shadow operations are state
convergent, if all blue operations are globally commutative. The first two lemmas remain valid under \PRCN\ since their proofs
remain unchanged. However, we have to change the third lemma since \RBCN\ archieves state convergence by requiring
all blue shadow operations to commute but \PRCN\ only needs any pair of unordered shadow operations
to commute. We apply the following changes to Lemma~\ref{lem:adjacentconvergent},
which results in a new lemma (Lemma~\ref{lem:poradjacentconvergent}).
\if 0
%\begin{lemma}\label{lem:canSwap}
% Given a legal serialization $O_i=(U,<_i)$ of R-Order $O=(U,\prec)$
% with shadow operations $u,v\in U$ such that $u<_iv$ and $u \not\prec v$ and
% there exists no $s$ such that $u<_{i}s<_{i}v$, and let $P=\{ p | p \in U \wedge p <_i u\}$
% and $Q=\{q | q \in U \wedge v <_i q\}$. The serialization $O_k=(U,<_k)$ where
%   \begin{itemize}
%     \item $\forall p,q\in P \cup Q: p<_kq
%       \Longleftrightarrow p<_iq$,
%   \item $\forall p \in P: p <_k v$,
%   \item $v<_k u$,
%   \item $\forall q \in Q: u <_k q$
%  \end{itemize}
%  is a legal serialization of $O$.
%  \end{lemma}
% \begin{lemma}\label{lem:adjexists}
% Given a legal serialization $O_i=(U,<_i)$ of R-Order $O=(U,\prec)$, if
% $\exists u,v\in U$ such that $u<_i v$ and $u\not\prec v$,
% let $U' = \{u,v\} \cup \{q|u<_i q\wedge q<_i v\}$, then $\exists r, s \in U'$ such that $r<_i s \wedge$ $r\not\prec s$ $\wedge \not\exists p \in U': r<_i p\wedge p <_i s$.
% \end{lemma}
\fi

\begin{lemma}\label{lem:poradjacentconvergent}
Assume $O_i=(U,<_i)$ and $O_j=(U,<_j)$ are both legal
serializations of R-Order $O=(U,\prec)$ that are identical except for
two adjacent operations $u$ and $v$ such that $u<_iv$ and $v<_ju$ and that $u$ and $v$ commute. Then
$S(O_i)=S(O_j)$.
\end{lemma}

The above lemma asserts that two legal serializations that differ
in the order of exactly one pair of adjacent shadow operations are state 
convergent, if the two operations commute. We introduce the proof 
of Lemma~\ref{lem:poradjacentconvergent} as follows.

\noindent{\bf Proof:} Let $P$ and $Q$ be the greatest common prefix and suffix
of $O_i$ and $O_j$, respectively.  Further, let $S_P=S_0(P)$,
$S_{uv}=S_P+u+v$, and $S_{vu}=S_P+v+u$. By the definition of operation commutativity,
$S_{uv} = S_{vu}$. It then follows the definition
of deterministic state machine that $S_{uv}(Q) = S_{vu}(Q)$. 
By the definition of legal serialization (Definition~\ref{def:legalserial}),
the final state reached by sequentially executing operations in $O_{i}$ 
against $S_0$ according to $<$ is equal to the final state obtained by 
sequentially applying operations in $Q$ against $S_{uv}$ according to $<$, namely $S_0(O_i)=S_{uv}(Q)$. By a
similar argument, we know $S_0(O_j)=S_{vu}(Q)$. Finally, we have $S_0(O_i)=S_0(O_j)$.
\qed

After adapting these lemmas from Chapter~\ref{chapter:redblue} to \PRCN, we use
them to construct the proof of the state convergence theorem (Theorem~\ref{theorem:convergence}) as follows:

\noindent{\bf Proof:}
($\Leftarrow$:) We first show that if for any pair of non-commuting operations of $\mathscr{S}$,
a restriction between this pair of operations is in $R$, then the \PRCAJ\ system $\mathscr{S}$ is convergent. 
To prove this, it is sufficient to show that any pair of legal serializations of their underlying R-Order $O$ 
is state convergent. Let $O_{i}$ and $O_{j}$ be two legal serializations of $O$. Let $S$ be the initial state.
There are two cases to consider:

{\bf Case 1:} $O_i = O_j$.  The underlying deterministic state
machine ensures that $S(O_i)=S(O_j)$.

{\bf Case 2:} $O_i \neq O_j$, in which case $\exists u,v\in U$
such that $u<_i v$ and $v<_j u$. Since both $O_i$ and $O_j$ are
legal serializations of $O$, it follows that $u\not\prec v$ and
$v\not\prec u$. It then follows from Lemma~\ref{lem:adjexists} that we
can find an adjacent pair of operations $r, s$ in both $O_i$ and $O_j$ such
that $r<_i s \wedge s<_j r \wedge r\not\prec s \wedge s\not\prec r$. 
We construct a new serialization $O_{i+1}$ by duplicating $O_i$ but swapping the 
order of $r$ and $s$ in $O_{i+1}$, i.e., $r<_i s \wedge s<_{i+1} r$. By Lemma~\ref{lem:canSwap}, 
$O_{i+1}$ is also a legal serialization of $O$. It then follows from the hypothesis that $r$ and $s$
commute and from Lemma~\ref{lem:adjacentconvergent} that $O_i$ and $O_{i+1}$ are convergent.

If $O_{i+1} \neq O_j$, we continue the construction by finding
an adjacent pair of operations whose order is different in $O_{i+1}$, $O_j$. By swapping
the two operations, we obtain another legal serialization $O_{i+2}$. We can then continue to swap
all such adjacent pairs until the last constructed serialization
is equal to $O_j$. This is achievable since at every step the number of operation pairs in the corresponding newly constructed legal serialization 
whose orders are different in $O_j$ decreases.
At the end, the construction process results in a chain of legal
serializations where the first one is $O_i$ and the last is $O_j$, and any consecutive pair of legal serializations
is identical except for the order of an adjacent pair of elements. It then follows
Lemma~\ref{lem:adjacentconvergent} that every consecutive pair of
serializations in the chain is state convergent, thus $S_0(O_i) = S_0(O_j)$.


\noindent ($\Rightarrow$:) (Proof by Contradiction.) We show that if a \PRCAJ\ system $\mathscr{S}$ 
with a restriction set $R$ is convergent, then for any pair of non-commuting shadow operations, there must exist
a restriction confining the two operations in $R$. Since $\mathscr{S}$ is convergent, we know that for any R-Order of $\mathscr{S}$, any pair of legal serializations of that R-Order are convergent. We assume by contradiction that there exist two
shadow operations $u, v$ such that they don't commute and $r(u, v) \not\in R$. By the definition of commutativity, there exists
a state $S$ such that $S+u+v \neq S+v+u$. We can find a state $S_0$ and a legal serialization of $\mathscr{S}$, $O_1(P, <)$, such that $S_0(P) = S$. Then,
we can construct a R-Order $O(U, \prec)$, where 

\begin{itemize}
\item $U = P\cup\{u, v\}$;
\item for any pair of operations in $P$, $m$ and $n$, $m<n \Longleftrightarrow m\prec n$;
\item for any operation $m$ in $P$, $m \prec u$ and $m \prec v$.
\end{itemize}  

$u, v$ are the maximal elements of $O$. It follows from the definition of causal legal serialization (Definition~\ref{def:causalserial}) that
we can construct two legal serializations $L_1$ and $L_2$ of $O$ such that $L_1 = P + u + v$ and $L_2 = P + v + u$. 
As $S_0(L_1) = S_0(P + u +v)$, $S_0(L_1) = S_0(P) + u +v$. It follows from $S_0(P) = S$ that $S_0(L_1) = S+u+v$. By a similar argument, $S_0(L_2) = S+v+u$.
It then follows from $S+u+v \neq S+v+u$ that $S_0(L_1)\neq S_0(L_2)$. As $L_1$ and $L_2$ are not convergent, $\mathscr{S}$ is not convergent. Contradiction is
found.
\qed

