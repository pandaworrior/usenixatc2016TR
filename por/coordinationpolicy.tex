\subsection{Coordination protocols}
\label{ch:por:sect:coordination}
In this part, we present the two coordination techniques that
we currently support in \coordtool\ and concrete
scenarios where these mechanisms are more adequate. We leave to
the future work the implementation of more techniques so that
programmers have more choices for their various tradeoff points. 
The two implemented protocols are namely, {\tt symmetry (Sym)} and {\tt asymmetry (Asym)}.
Given a restriction $r(u, v)$ between two operations $u$ and $v$, the 
{\tt symmetry} protocol requires both $u$ and $v$ to contact each other
for establishing an order between them. Unlike this, an {\tt asymmetry} protocol
provides different treaties for $u$ and $v$ by
only requiring $u$ (or $v$) to inform the counterpart operation in the restriction $v$ (or $u$) 
about its existence, while allowing $v$ (or $u$) to be fast executed without coordination 
if no $u$ (or $v$) operations are running simultaneously. We illustrate the two protocols as follows:

\paragraph{{\tt Sym:}} This protocol requires us to set up a centralized counter service, 
which maintains a counter for every shadow operation, i.e., $c_{u}$ and $c_{v}$,
and serializes reads and writes to these counters. Every such counter represents the total number of the corresponding
operation have been accepted by the underlying system. Additionally, every replica at different
data centers maintains a local copy of the counters, each of which represents the number of the corresponding
operation have been observed by that replica. Initially all local copies as well as the global counters have 
all values set to zero. Whenever an $u$ operation is received by a replica, 
that replica would contact the counter service to increase the
corresponding counter $c_{u}$ and get a fresh copy of the counter maintained
for $v$. Upon receiving the reply from the counter service
that replica can then compare the value of $c_{v}$ with its local copy.
If they are the same, then the replica can execute $u$ without waiting. If the value is greater than
the local copy, the local execution can only take place when all missing $v$ operations have been locally replicated.
The same procedure is also applied to $v$, vice versa. After replicating operations, 
the {\tt cleanUp} is called to bring the local copy of the counters to be up-to-date.
In order to make the counter service fault tolerant, we leverage the Paxos-like state machine replication technique (BFTSmart~\cite{Bessani2014SMR}) to 
replicate the service state in a few geo-locations.


\paragraph{{\tt Asym:}} Unlike the above centralized solution, the core of the asymmetry protocol
implements distributed barrier and operates via a decentralized manner as follows. Assume, for
simplicity that $u$ is the barrier. In this case whenever
a replica $r$ receives an operation $u$ it would have to enter the barrier,
and contact all other replicas to request this. This requires all
replicas in the system to stop processing $v$ operations.
After all replicas acknowledge the entrance of $r$ in the barrier for $u$, $r$ can execute the operation, and then notify all replicas 
that it has left the barrier (while at the same time propagating the effects
of the operation $u$ it has just executed).
Such a coordination strategy might incur
in a high overhead; however, it might be interesting when one of
the two operations in the restriction is rarely submitted to the system.
For instance, in the auction example, {\tt closeAuction'} is
a candidate for being used as barrier, since 
{\tt placeBid'} dominates the operation space.


