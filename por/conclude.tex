\section{Limitations and future work}
\label{ch:por:sect:limit}

While adapting applications to use \PRCN\ and \coordtool\ significantly outperforms
the usage of \RBCN\ and \tool, there are several interesting avenues for future work:

First, there might exist some restrictions which are not necessary to be symmetry. We have not
explored the existence of asymmetry restrictions, and not assessed the impact of having asymmetry
restrictions on the coordination cost. We leave this exploration to our future work.

Second, the adoption of \PRCN\ requires the programmer to manually execute the sequence
of algorithms for analyzing possibility of diverging state or violating invariants, in order to
obtain the minimal set of restrictions. To free this programming burden, we plan to develop
an automatic tool, which implements all these algorithms.

Third, the current implementation of \coordtool\ only embraces two different coordination
protocols, each of which is suitable for a certain workload. In future, we plan to incorporate
more efficient coordination protocols into \coordtool\ so that the programmer can make a better
choice.

Fourth, we plan to plugin an agent into \coordtool, which dynamically measures
the frequencies of different operations and makes runtime decisions on switching from
a protocol to another cheaper one.

\section{Conclusion}
\label{ch:por:sect:conclude}
In this chapter, we proposed a research direction for building fast and consistent
geo-replicated system that employs a minimal amount of coordination in
order to achieve both invariant preservation and state convergence. 
To this end, we first defined a new generic consistency model called \PRCN, which maps
consistency requirements to fine-grained restrictions over pairs of operations. Second,
we developed a static analysis to infer for a given application
a minimal set of restrictions for ensuring the two previously mentioned properties, in which
no restrictions can be removed or no new restrictions need to be added. Third, we built an efficient
coordination service called \coordtool, which offers two different coordination protocols
suiting for different workloads. Our evaluation of running RUBiS with different
setups shows that the joint work of \PRCN\ and \coordtool\ significantly
improves system performance of geo-replicated systems.
