\subsection{Case studies}
\label{ch:por:sect:casestudies}
\begin{table}[t!]
\centering
\begin{tabular}{|l|l|l|}
\hline
App & RedBlue consistency & PoR consistency\\
\hline
\hline
RUBiS & $r(registerUser, registerUser)$ & $r(registerUser, registerUser)$ \\ 
      & $r(storeBuyNow, storeBuyNow)$ & $r(storeBuyNow, storeBuyNow)$\\
      & $r(placeBid, placeBid)$ & $r(placeBid, closeAuction)$\\
      & $r(closeAuction, closeAuction)$ & \\
      & $r(placeBid, closeAuction)$ & \\
      & $r(registerUser, storeBuyNow)$ &\\
      & $r(registerUser, placeBid)$ & \\
      & $r(registerUser, closeAuction)$ &\\
      & $r(storeBuyNow, placeBid)$ & \\
      & $r(storeBuyNow, closeAuction)$ & \\
\hline
\end{tabular}
\caption{Restrictions}
\label{tab:restrict}
\end{table}

We apply the analysis (Algorithm~\ref{alg:mrdiscover}) to a completed version of RUBiS, which
implements a {\tt closeAuction} operation for declaring auction winners, to
identify a set of restrictions comprising a minimal amount of coordination without
sacrificing either state convergence or invariant preservation. This subsection
reports our experience on conducting such analysis and the final static result we
obtained.

\noindent\paragraph{State convergence:} As we deploy RUBiS alongside \tool,
all shadow operations generated at runtime commute w.r.t each other and
there is no need to restrict any pair of shadow operations. The final output of
the state convergence restriction discovery method (Algorithm~\ref{alg:scdiscover}) is an empty restriction set.

\noindent\paragraph{Invariant preservation:} By analyzing the source code, we determined
four invariants of the completed version of RUBiS, namely, (a) identifiers 
assigned by the system are unique; (b) nicknames chosen by
users are unique; (c) item stock must be non-negative; and (d) the auction 
winner must be associated with the highest bid across all accepted bids. We continued by
performing the {\tt I-conflict set} analysis (Algorithm~\ref{alg:iconflictassess}) against all RUBiS shadow operations. With regard
to the first invariant, since we take advantage of the coordination-free unique identifier generation method
offered by \tool, no {\tt I-conflict sets} were found for violating it. In contrast, 
for the remaining three invariants, we identified
the following {\tt I-conflict} sets: 

\begin{itemize}
 \item$\{registerUser', registerUser'\}$. The (b) invariant would
be violated if the two operations proposed the same $nickname$ and were submitted to different sites simultaneously;
 \item $\{storeBuyNow', storeBuyNow'\}$. The (c) invariant would be violated if both operations simultaneously
deducted a positive number from $stock$ while $stock$ was not enough; 
 \item $\{placeBid', closeAuction'\}$. The (d) invariant would be violated if both operations were submitted at the same
time to different sites plus $placeBid'$ carried a higher bid than all accepted bids.
\end{itemize} 
Each of the three above {\tt I-conflict sets} covers a class
of violating executions of the respective invariant. To eliminate these
violations, according to Algorithm~\ref{alg:isdiscover}, we added the following restrictions, namely,
$r(registerUser', registerUser')$, $r(storeBuyNow', storeBuyNow')$ and $r(placeBid', closeAuction')$, 
which are summarized in Table~\ref{tab:restrict}. This set is a minimal restriction
set since it is sufficient to ensure the two important properties while none of these restrictions
can be removed. However, compared to the PoR consistency solution, replicating RUBiS
via RedBlue consistency would require more restrictions, since the definition states that
all non-invariant safe shadow operations must be red (strongly consistent), i.e., the four shadow operations presented
in the above list must be restricted in a pair-wise fashion, as shown in Table~\ref{tab:restrict}. 
\if 0
We have previously introduced an auction service
as a motivating example for the limitations of hybrid consistency models such as
\RBCN. We \changebars{are able to compose an invariant-violating
execution with three concurrent operations touching a shared auction, 
namely, a {\tt close\_auction} operation and
two {\tt place\_bid} operations whose associated bids are higher than
all accepted bids.}{remind the reader that such a service supports two operations, in 
particular a {\tt place\_bid} operation to submit a new bid for an
auction, which is accepted provided that the target auction is not closed; and a {\tt close\_auction}
operation which closes the auction while at the same time declares a winner. 
An application-specific invariant is that the declared winner must be the user
that issued the highest accepted bid (while the auction was not closed).}
\changebars{According to the definition in Section~\ref{sect:intro},
this execution is not minimal since the composition of a {\tt place\_bid}
and {\tt close\_auction} operation is already sufficient to trigger the violation. Thus,
we refine the execution by removing one of the {\tt place\_bid} operations from it.}{} 
As a way to \changebars{preserve}{enforce} 
the \changebars{}{application-specific }invariant of this service under \PRCN\ 
a restriction \changebars{must}{can} be created between \changebars{any pair of a}{operations of type} {\tt place\_bid} and 
\changebars{a}{operations of the type} {\tt close\_auction}\changebars{ operation,
i.e., $r({\tt place\_bid}, {\tt close\_auction})$}{}. 
The advantage of this approach is that \changebars{}{ the instances of}{\tt place\_bid} \changebars{operations}{}
have no restrictions among themselves, which implies that no coordination mechanism is required when executing \changebars{them
concurrently}{several 
instances of this type of operation}. \changebars{Given the fact that these
operations}{ which, incidentally,} are the most common in
such a service\changebars{, performance will be dramatically improved}{}.
\fi
\if 0
\paragraph*{Banking Service} Another interesting example is an
online banking service, that supports operations such as {\tt deposit} and 
{\tt withdraw} which respectively, allows a user to add or remove a given amount to or from
the current bank account balance, while a third operation\changebars{}{ type denominated}
{\tt accrueinterest}, updates the balance value of an account by taking into
consideration its current balance and a given interest rate. \changebars{We further assume}{In this service
one can imagine that there is } an application-specific invariant which states
that account balance values must be non-negative.

In this particular example, to ensure state convergence, \changebars{}{operations
of type}{\tt accrueinterest} operations must be executed with coordination in relation 
to \changebars{}{operations of type}{\tt deposit} or {\tt withdraw} operations\changebars{, i.e.,
creating two restrictions $r({\tt accrueinterest}, {\tt deposit})$
and $r({\tt accrueinterest}, {\tt withdraw})$, as multiplication does not
commute with addition or subtraction.}{ In \PRCN\ this is modeled
by creating two restrictions respectively, between  {\tt accrueinterest} and 
{\tt deposit} and between {\tt accrueinterest} and {\tt withdraw}.}
However to ensure the application-specific invariant, special care has to
given to \changebars{}{operation of the type }{\tt withdraw}\changebars{ operations, as
two concurrent {\tt withdraw} operations may drive the balance value to negative}{}. 
\changebars{As a result}{In fact}, \changebars{}{all instances of }these
operations have to be coordinated among themselves. To capture this 
fact one has to add an additional restriction \changebars{}{to order operations of type
{\tt withdraw}, i.e.,}$r({\tt withdraw}, {\tt withdraw})$. \changebars{}{This approach allows 
operations of type {\tt deposit}
to be executed without any form of coordination
among themselves.}
\fi
